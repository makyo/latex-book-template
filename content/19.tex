\chapter{Kuracisto}

Ankoraŭ du jarojn studadis Karlo en la malsanulejoj de Bruselo, sed li revenadis hejmen por Kristnasko, por Pasko, por la someraj monatoj kaj ankoraŭ eĉ pli ofte. La edziniĝo de lia fratino Heleno estis unu el tiuj bonaj kaŭzoj por libertempo.

Ĉe tiu festo li la unuan fojon montriĝis oficiale kun sia fianĉino Alico Palam. Estis tre ĝojiga tago.

Ĉe la fino de sia lasta studjaro, Karlo prezentis ĉe la Universitato sian tezon pri ``la influo de l'imagemo en la muskolaj malsanoj''. Sukcesinte kun honoro, li definitive revenis al sia hejma urbo.

Konsilate de sia patro, li luis ĉambraron sur la unua etaĝo de komforta domo tute proksime je la vendoplaco. Alico lin helpis por aranĝadi la ĉambraron kaj iliaj patrinoj sin okupis pri aĉeto de mebloj, tapiŝoj, kurtenoj, k.t.p.

Karlo deziris havi kelkajn artajn bildojn en sia hejmo. Alico donacis al li grandan kopion de la itala pentraĵo, antaŭ kiu ili ambaŭ haltis samtempe en la muzeo, kiam ili estis geknaboj.

Per la amikoj kaj la konatuloj de ambaŭ familioj Davis kaj Palam, Karlo ekakiris kelkajn klientojn. Sian tezon de doktoro de medicino li vendigis en la librobutikoj de la urbo kaj per tio plikonigis sian nomon, kvankam la tezo ne pli vendiĝis, ol lia poemlibro. Pro lia granda kvieteco kaj certeco, Karlo tre rapide plimultigis sian klientaron. Li efektivigis kelkajn mirindajn resanigojn de nervaj personoj, kaj li ĉiam pli kaj pli interesiĝis je tiu psikologia parto de l'kuracarto.

Li interesiĝis je publika higieno kaj faris kelkajn paroladojn pri hejmozorgado, pri ordo en ĉiutaga vivado, pri reguleco en manĝado kaj dormado, pri efikoj de alkoholdrinkado, k.t.p.

Post ses monatoj li estis preskaŭ la plej ŝatata kuracisto en la urbo kaj jam estis plu nenia kaŭzo por prokrasti lian edziĝon kun Alico Palam. Ambaŭ gejunuloj konsentis por ne havi bruan feston. Ili opiniis, ke oni ne edziĝas por la publiko. Laŭ ilia deziro la festo estis tute familia kaj intima, kaj kredeble pro tio speciale ĉarma. Sinjorino Palam havis tamen doloran momenton da plorado pensante, ke ŝi de nun ne plu havos kun si sian amatan filinon. Sed la feliĉo de Alico ŝin konsolis. Ŝi sukcesis iom forgesi tiun suferigan penson, ke ofte ies feliĉo kaŭzas la malfeliĉon de alia, kaj kio alportas rideton sur la lipojn de unu, ofte naskas larmojn en la okuloj de alia.

Post la edziĝa ceremonio, la feliĉaj geedzoj foriris por fari la kutiman vojaĝon.

\newpage

\section*{Demandoj}

\begin{enumerate}
    \item  Ĉu Karlo kelkafoje revenis hejmen?
    \item  Kiun tezon li prezentis?
    \item  Sur kiu etaĝo li luis ĉambraron?
    \item  Kiu helpis Karlon?
    \item  Kiuj objektoj troviĝas ĝenerale en salono kaj en manĝoĉambro?
    \item  Kion donis Alico al sia fianĉo?
    \item  Kies tezo ne multe vendiĝis?
    \item  Ĉu vi legis la tezon de S-ro Doktoro Corret pri la utileco de internacia lingvo por medicino?
    \item  Kio estas en la kapo?
    \item  Per kio oni vidas?
    \item  Nomu la diversajn partojn de la homa korpo.
    \item  Kiu mano estas ĝenerale pli lerta?
    \item  Kiel oni nomas viron, kiu ne povas vidi?
    \item  Kiel skribas la blinduloj?
\end{enumerate}
