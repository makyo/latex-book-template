\chapter{La hejmo}

Junaj gesinjoroj Davis estis nek riĉaj, nek malriĉaj, kaj havis plej ĉarman hejmon apud la urbeto. Tre simpla dometo, malgranda ĝardeno kun kelkaj arbetoj, jen ilia tuta hejmo, sed ĉio estis tre pura kaj delikata interne kaj ekstere. En la ĉambroj ĉio estis hela, la murpaperoj kaj la meblaro. Ĉie oni povis admiri kiel orde kaj elegante la delikataj manoj de Sinjorino Davis aranĝis ĉiujn aferojn.

La patro de Karlo-Teodoro estis ĉefkomizo en filio de la Nacia Banko kaj li ricevis tre bonan salajron. Ĉar nek li, nek lia edzino multe elspezis, li povis ĉiujare ŝpari sumon por ``siaj infanoj''.

Efektive Karlo ne ĝuis tre longe la gloron esti sola en la nova generacio de l'familio. Kiam li estis dujara kaj jam kuris tre bone tra la tuta domo, li trovis iun tagon bluokulan fratineton en la antikva lulilo, kie li kutimis dormadi antaŭ tre longe\ldots{}kiam li estis malgranda.

Aliaj kredeble estus iĝintaj tre ĵaluzaj, sed Karlo estis grandanima, kaj lia vizaĝo tuj montris larĝan rideton kaj afablan esprimon de protektemo. Petinte la permeson de sia patrino, li kisis dufoje la frunton de sia dormanta fratineto.

Karlo tre multe interesiĝis je la progresoj de sia fratineto Helenjo. Kiam ŝi ankaŭ povis kureti tra la domo, li fariĝis ŝia instruanto kaj gvidanto.

Karlo estis tre observema kaj ankaŭ tre entreprenema. Li penadis imiti ĉion interesan, kion li vidis aŭ pri kio li aŭdis. Uzante tablon kaj multajn seĝojn, li konstruis grandan ŝipon, kies bela flava kamentubo estis\ldots{}la paperkorbo de lia patro. Por Helenjo li ĉiam aranĝis komfortan sidejon per broditaj kusenoj forprenitaj el la salono.

Kiam Sinjorino Davis laboris ĉe sia kudromaŝino kaj Karlo ekfajfis kaj bruadis por anonci, ke danĝera ventego minacas la ŝipon, la timigata fratineto ekploris, tremante inter la broditaj kusenoj. Subite ĉio haltis: la kudromaŝino kaj la ŝipo. Helenjo baldaŭ retrankviliĝis sur la genuoj de l'patrino, konsolita, ĉu per ŝiaj dolĉaj kisoj, ĉu per la elokventaj klarigoj de Karlo, kiu venigis la ventegon nur por havi la okazon kuraĝe savi sian fratinon.

Supre en la domo estis longa subtegmenta ĉambro, en kiu kuŝis ĉiaj kestoj, korboj, seĝoj kaj malnovaj objektoj. Estis por Karlo vera paradizo. Kiam li trovis malfermita la pordon de l'ŝtuparo, li rapide rampis al la ŝatata ĉambrego. Tie li plezurege ĉirkaŭpromenis, malfermante la kestojn, palpante kaj malordigante ĉiujn aferojn. La ĝojo daŭris ĝis lia patrino aŭ la servistino, aŭdinte bruon, venis lin serĉi kaj rekondukis lin malsupren.

\newpage

\section*{Demandoj}

\begin{enumerate}
    \item  Ĉu gesinjoroj Davis estis riĉaj?
    \item  Kie ili loĝis?
    \item  Citu kelkajn meblojn, kiuj troviĝas en via hejmo?
    \item  Kion faris la patro de Karlo?
    \item  Ĉu li multe elspezis?
    \item  Ĉu Karlo estis ĵaluza?
    \item  Kiel oni nomis lian fratinon?
    \item  Per kio Karlo konstruis ŝipon?
    \item  Sur kion li sidigis Helenjon?
    \item  Per kio oni kudras?
    \item  Kiu estas la plej granda firmo por la fabriko de kudromaŝinoj?
    \item  Kiu ĉambro estis en la supro de la domo?
    \item  Kio estis en ĝi?
    \item  Ĉu Karlo ŝatis iri tien?
    \item  Kion li faris en tiu ĉambro?
    \item  Kiu kondukis lin malsupren?
    \item  Ĉu estas lifto en via domo?
    \item  Ĉu vi loĝas sur la unua etaĝo?
    \item  Kiom da loĝantoj estas en via urbo?
\end{enumerate}
