\chapter{La servistino}

La geavoj de Karlo konservis tre longe unu servistinon, kiun ĉiuj en la familio tre amis pro ŝia fideleco kaj diskreteco. Kiam la juna Sinjoro Davis edziĝis, la servistino iris en la novan hejmon por ĉion aranĝi. La juna edzino tiom plaĉis al ŝi, ke ŝi tie restis; kaj post la naskiĝo de Karlo, ŝi decidis, ke ŝi neniam foriros.

Ŝia nomo estis Anjo. Ŝi havis pintan nazon kun du malgrandaj grizbluaj okuloj kaj vangoj ĉiam tre ruĝaj, kiel pomoj. Sian blankan kufon ŝi preskaŭ neniam demetis, sed ofte okazis, ke Karlo fortiris ĝin por vidi la koloron de ŝiaj haroj. Ĉiufoje li ricevis pro tio vangofrapon kaj tuj poste kison.

En la domo Anjo faris ĉion; ŝi estis samtempe kuiristino kaj ĉambristino, sed la domo estis tiel malgranda, ke ŝi facile plenumis ambaŭ taskojn. Sinjorino Davis ŝin ofte helpis kaj faris ĉiujn kudrolaborojn. Anjo sentis en si tre grandan admiron al la juna dommastrino pro ŝia granda delikateco en ĉio.

Kelkafoje Karlo ricevis permeson iri frumatene al la legomvendejo kun Anjo. Ŝi portis du grandajn korbojn kaj li malpezan retsakon. Post dekkvinminuta marŝado ili alvenis al la publika placo. Multaj ĉevaloj kaj veturiloj staris apud la trotuaroj. En la mezo de la placo, ĉirkaŭ la monumenta fontano, sidis la gevendistoj kun siaj tabloj kaj korbegoj plenaj je brasikoj, terpomoj, fromaĝoj, fruktoj ĉiuspecaj. Anjo diskutadis kun multaj virinoj kaj fine aĉetis diversajn legomojn kaj grandan pecon da butero. Kelkafoje ŝi metis ion en la retsakon de Karlo. Poste tre rapide ili hejmen revenis.

Kelkafoje Karlo helpis Anjon meti la telerojn sur la tablon. La forkojn, kulerojn kaj tranĉilojn li ankaŭ alportis, sed neniam la glasojn, ĉar Anjo opiniis la aferon tro danĝera.

Iam li rompis tason, ĉar li provis porti tri samtempe. Alkuris Anjo ĉe la bruo\ldots{}``Estas nenio,'' ekkriis Karlo, ``mi ne vundiĝis.'' Anjo lin longe riproĉis, sed Karlo opiniis, ke estus multe pli saĝe, se ŝi irus tuj serĉi alian tason.

Kun Karlo, Anjo estis ĉiam gaja kaj ŝercema; sed kun Helenjo, ŝi estis multe pli dolĉa kaj eĉ ŝajnis iom malĝoja rigardante ŝin. Ŝi prenis ŝin delikate en siaj brakoj kaj movis la kapon penseme. Ĉiuj same ŝajnis melankolie dolĉiĝi apud Helenjo.

Karlo tre amis sian fratineton, sed ne kuraĝis multe ludi kun ŝi, ĉar ŝi ŝajnis timi la bruon kaj tre ofte ekploris. Li opiniis, ke la knabinoj estas delikataj estaĵoj, kiuj bezonas multe da zorgoj.

\newpage

\section*{Demandoj}

\begin{enumerate}
    \item  Kiel nomiĝis la servistino de S-ino Davis?
    \item  Ĉu ŝi estis diskreta?
    \item  Kiu estis la koloro de ŝiaj vangoj?
    \item  Ĉu ĉiuj pomoj estas ruĝaj?
    \item  Kio estas en la mezo de pomo?
    \item  Citu kelkajn fruktojn.
    \item  Kiun vi preferas?
    \item  Ĉu Karlo kelkafoje tiris la kufon de Anjo?
    \item  Kien iris Karlo kun la servistino?
    \item  Kion li portis?
    \item  Kiuj personoj estis sur la publika placo?
    \item  Kion ili faris?
    \item  Citu kelkajn legomojn?
    \item  Kion aĉetis Anjo?
    \item  Ĉu Karlo helpis Anjon por prepari la tablon?
    \item  Kiom da tasoj rompis Karlo?
    \item  Kion li respondis al la riproĉoj de Anjo?
    \item  Ĉu Helenjo estis amata de sia frato?
    \item  Kion opiniis Karlo pri Helenjo?
    \item  Kiujn ludilojn preferas la knabinoj?
    \item  Ĉu kelkafoje Helenjo ploris?
    \item  Kiam oni ploras?
\end{enumerate}
