\chapter{La sekreto de Karlo}

Ĉiudimanĉe la familio Davis iris al la ĉefpreĝejo, matene je duono post la deka. Ĝi estis tre malnova katedra preĝejo, ekkonstruita en la centjaro de la imperiestro Karlo Granda.

Dum la orgeno eksonis laŭtege, ĉe la komenco de l'diservo, Karlo sidiĝinte apud siaj gepatroj okule serĉadis iun sur la benkoj antaŭ si. Li estis maltrankvila ĝis li rekonis blondan hararon kun blua rubando, kiu ĉiudimanĉe aperis tie. Lumigata de sunradio falinta tra la koloraj vitraĵoj, la blonda hararo brilis kvazaŭ oro inter la nigriĝintaj kolonoj de l'antikva templo. Tiu vidaĵo trankviligis la animon de Karlo, kaj post la preĝoj li ne movis plu siajn okulojn dum la tuta predikado de l'pastro. Post la diservo Karlo kutime zorgis por vidi la junulinon elirantan; kaj ĉiufoje li konvinkiĝis, ke ŝi havas belege bluajn okulojn.

Iam promenante tra l'artmuzeo kun amiko, li vidis ŝin admire starantan antaŭ pentraĵo de itala majstro. Li kompreneble haltis antaŭ la sama. Ŝi tre ruĝiĝis; sed de tiu tempo, ĉiudimanĉe ili ofte ekrigardis unu la alian elirante el la ĉefpreĝejo. Ŝi ĝenerale estis sola kun sia patrino, sed kelkafoje ili salutis kaj alparolis alian sinjorinon, kiu eliris el preĝejo kun kvar gefiloj. Karlo iam aŭdis ŝin diri al ili: ``Vi do venos ludi en la Nacian Parkon hodiaŭ posttagmeze, ĉu ne?''

Post iom da pripensado Karlo decidis, ke li ankaŭ iros al la parko posttagmeze. Tie li havis mirindan ŝancon, ĉar promenante apud la ludkampo, li dufoje havis okazon redoni al ŝi pilkon falintan sur la vojon aŭ perditan sub arboj. Ŝia rideto kaj danko lin strange konfuzis. La postan dimanĉon, li sekvis ŝin de malproksime por vidi, kie ŝi loĝas. Ĝi estis ekster la urbo sur strato bordita de arboj kaj ĝardenoj. Ŝi eniris en kastelforman dometon kun belega florĝardeno plena je rozoj.

Karlo jam ofte rimarkis tiun domon antaŭe, sed ŝajnis al li, ke ĝis nun ĝi ĉiam estis fermita; kaj li kredis, ke ĝi estas neluita. Kredeble ŝia familio ĵus luis aŭ aĉetis la domon.

De tiam Karlo ofte vagadis ĉirkaŭ tiu loko, esperante ŝin ŝance ekvidi. Unu vesperon li vidis ŝin promeni kun ŝia patro en la ĝardeno. Alian fojon li aŭdis ŝin kanti tra malfermita fenestro. Tio lin feliĉigis por multaj tagoj. Jam lia fratino kaj liaj kolegoj ĉe la gimnazio multe miris pri lia ofta revemeco.

Unu tagon li trovis ŝian domon tute fermita: Teruran baton li ricevis en la koro. Reveninte la morgaŭan tagon li same vidis. Ŝi eble forveturis kun la tuta familio. Sed kien? En la ĝardeno kelkaj nigraj birdoj pepante interbataletis. Karlo jam kredis, ke lia tuta vivo estas ruinigita.

Li decidis diri al Janko sian sekreton kaj peti lian helpon.
Rakontu skribe la supran ĉapitron.

\newpage

\section*{Demandoj}

\begin{enumerate}
    \item  Kion faris la familio Davis dimanĉe?
    \item  Ĉu la katedra preĝejo estis nova?
    \item  Kiun renkontis Karlo en la muzeo?
    \item  Kion diris la fraŭlino iun tagon?
    \item  Kion faris Karlo?
    \item  Kie loĝis la fraŭlino?
    \item  Nomu kelkajn florojn kaj diru, kiun vi preferas?
    \item  Kiuj birdoj estis en la ĝardeno?
\end{enumerate}
