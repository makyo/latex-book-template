\chapter{La profesoro de historio}

La morgaŭan tagon, la klaso de Karlo ne havis lecionon de greka lingvo, ĉar la profesoro forestis, malsana. Janko, Servetti kaj Karlo opiniis, ke ĉi tiun tagon la ŝanco ilin favoras. Sed restis la profesoro de historio, kiu sendube ion diros. Kiam, je la dekunua kaj dek minutoj precize li eniris la klason, la tri kune ektimis kaj tute speciale Karlo. Sed S-ro Jehmann tuj komencis sian kurson pri la regado de l'imperiestro Trajano kaj diris nenion pri la forvago de la tri knaboj. Karlo tiom pli timis la petitan viziton je duono post la kvara.

S-ro Jehman estis tute ne riĉa kaj havis kvar junajn infanojn. Lia edzino estis bela Italino el Bolonjo. Ŝi verkis ofte artikolojn por italaj revuoj kaj gazetoj. S-ro Jehman ĉiuprintempe veturis Italujon kun ŝi, ĉe la paska libertempo. Ambaŭ estis artistoj. Ili konis en Italujo ĉiujn preĝejojn, templojn kaj palacojn, kie estas majstraj pentraĵoj.

La lastan jaron, kiam ili revenis, S-ino Jehman trinkis glason da glacia limonado ĉe la stacio en Milano. Neniam oni sciis, ĉu tro malvarma, ĉu venenita ĝi estis. Sed post apenaŭ unu horo ŝi mortis. Estis terure. S-ro Jehman kvazaŭ bedaŭris, ke li ne ankaŭ trinkis limonadon. Sed li havis kvar infanojn. Reveninte hejmen, li devis klarigi al ili, ke sian patrinon ili neniam revidos. Nun la avino loĝadis tie kaj zorgis pri la infanoj, dum la patro instruadas. Sed S-ro Jehman perdis sian ĝojon; li nun malmulte parolis, malofte ridetis. Lia vizaĝo rapide maljuniĝis. Nur liaj brilaj okuloj montris ankoraŭ pli da bonkoreco kaj kompatemo.

Karlo pensis pri ĉio ĉi, posttagmeze, irante al lia domo. Li sonorigis ĉe la pordo. Servistino lin kondukis al la laborĉambro de S-ro Jehman.

La profesoro leviĝis kaj premante la manon de Karlo: ``Davis,'' li diris, ``mi ĝojas vin vidi tie ĉi, sidiĝu. Vi demandis min antaŭ kelkaj tagoj pri la kluboj kaj societoj en Romo ĉe la tempo de Cicero. Jen estas du libroj, en kiuj vi trovos informojn pri tio; mi krajone notis la interesajn paĝojn. Mi jam alportis ilin por vi hieraŭ en la liceon.'' Karlo estis tre konfuzita. ``Sinjoro,'' li diris, ``mi dankas vin, vi estas tro bona al mi\ldots{}Hieraŭ mi remadis sur la rivero dum via leciono. Mi petas vian pardonon, mi certe estas tre kulpa . . .'' — ``Ne plu parolu pri tio,'' diris S-ro Jehman, ``mi ja rimarkis, ke Janko, Servetti kaj vi forestis, kaj vidante la belan veteron, mi iom suspektis la veron. Mi scias, ke forvagado estas tre poezia afero kaj dolĉaĵo por liceanoj. Tamen atentu! Perdinte lecionojn, vi estos malhelpata en via studado kaj bezonos hejme laboradi pli longe. Nu, prenu la librojn kaj revenu iam min viziti. Mi ĉiam plezure kunparolos kun vi kaj vin helpos laŭpove.'' Kortuŝite eliris Karlo kaj ekpensis, ke li neniam vidis pli bonan manieron puni lernanton.

\newpage

\section*{Demandoj}

\begin{enumerate}
    \item  Ĉu Karlo havis — de greka — la morgaŭan tagon? Ne, ĉar la — estis malsana.
    \item  S-ro Jehman diris ion al la — ? Ne, li diris — ?
    \item  — da infanoj havis la profesoro? — havis
    \item  Kie — lia edzino? Ŝi mortis — Milano.
    \item  — donis la — al Karlo? Du librojn.
    \item  Ĉu la profesoro — Karlon? —, li — punis —.
\end{enumerate}
