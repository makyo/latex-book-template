\chapter{La ideoj de Karlo}

Karlo estis tre pensema kaj jam havis precizajn ideojn pri multe da aferoj. Dum la aliaj infanoj malfacile komprenis geografion, li tre bone prezentis al si la formon de la mondo: Post la montoj estas maroj, teroj kaj denove montoj ĝis la du ekstremaĵoj de la mondo. Tie sendube estas muro aŭ barilo, kiel sur ia ponto, por malhelpi, ke la personoj falu en abismon.

Karlo tre deziris iri iam al unu el tiuj ekstremaĵoj de la mondo. Certe estus strange vidi nenion plu antaŭen: nur la bluan ĉielon supre, kaj sub ĝi\ldots{}nenion. Li tre miris kien falus iu homo, kiu estus forsaltinta de l'ekstremaĵo de l'mondo. Se li posedus la aeroplanon de la riĉa sinjoro, kiun konas lia patro, li provus flugi malsupren sub la mondon kaj reveni supren ĉe la alia flanko. Li miris kiel estas la subaĵo de la mondo.

Karlo demandis sian patron pri tio. S-ro Davis estis tre amuzata de la ideoj de Karlo kaj klarigis al li, ke la mondo estas kvazaŭ grandega oranĝo, rondiranta ĉirkaŭ la suno, kiu estas ankaŭ tre granda globo. Ĉiuj steloj kaj la luno estas ankaŭ globoj. La patreto klarigis al Karlo, ke la homoj vivas sur la tuta terglobo. Sed Karlo ne povis kredi tion kaj pensis, ke certe lia patro eraras. Kiel la homoj povus stari sub la tero?

Tiu demando lin multe priokupis, kaj li ŝajnis pripensadi dum la tuta tagmanĝo post la respondo de sia patro. Lia patrineto promesis ke, kiam li scios bone legi, ŝi aĉetos por li ilustritan libron pri tiuj aferoj.

Karlo havis grandan admiron al ĉiuj metiistoj, al ĉiuj manlaboristoj. Kun sia patrino li jam vizitis la paniston kaj vidis lin bakanta la panojn. Li vidis la ŝufariston enpikanta najlojn en la ŝuojn per martelo. Li eĉ vidis ie en la urbeto grandan ĉambron, kie multegaj junulinoj fabrikas ĉapelojn por sinjorinoj. Ili aranĝis rubandojn kaj plumojn sur la ĉapelo. Pasante apud granda hotelo, ili vidis, tra malaltaj fenestretoj, grandan kuirejon, kie multaj viroj, kun blankaj antaŭtukoj kaj ĉapoj, movis pladojn kaj kaserolojn kun granda bruo. Odoro de rostaĵo alvenis al la nazeto de Karlo.

En la stratoj ĉiu ŝajnis rapidi: knabo puŝis veturileton ŝarĝitan je pakaĵoj; maljuna lama viro kriante vendadis ĵurnalojn. S-ino Davis klarigis al sia filo, ke ĉiuj homoj laboras por gajni monon kaj aĉeti siajn vestojn kaj nutraĵojn kaj tiujn de sia familio.

Ankaŭ tio multe pensigis Karlon. ``Kion faras patreto?'' li demandis. La patrineto kondukis lin al la oficejo de sia edzo; kaj li vidis sian patron skribanta sur granda librego ĉe alta tablo. Karlo opiniis, ke lia patro havas tre enuigan profesion kaj li decidis, ke li mem preferos fariĝi kuiristo en hotelo aŭ veturigisto.

\newpage

\section*{Demandoj}

\begin{enumerate}
    \item  Kiel Karlo prezentis al si la formon de la mondo?
    \item  Ĉu li bone komprenis la geografion?
    \item  Kion li studos en la lernejo, krom la geografio?
    \item  Kio estas aeroplano?
    \item  Kiu fabrikas la panon?
    \item  Kaj la kukojn?
    \item  Kion oni povas aĉeti ĉe la spicisto?
    \item  Kiom kostas unu funto da pano?
    \item  Kion oni metas sur la ĉapelojn por sinjorinoj?
    \item  Kiu profesio plej plaĉis al Karlo?
\end{enumerate}
