\chapter{Ekzameno}

La matenon de la ekzamena tago Karlo alvenis tre frue en la korton de la liceo. Li portis belan ledan sakon, tute novan, kiun lia patro ĵus donacis al li. En ĝi estis papero kaj skribilaro kun inkujo, plumoj kaj krajonoj en blua skatolo. En la korto estis jam kelkaj knaboj, sed neniu konata de Karlo. Ĉiumomente alvenis kelkaj aliaj. Fine Karlo ekvidis kamaradon, kiu ĉeestis du jarojn kun li la lernejon de S-inoj Linar. Lia nomo estis Henriko Belnett.

Karlo tuj iris saluti Belnett kun granda ĝojo, ke li fine trovis iun konaton. Belnett jam de unu jaro estis lernanto ĉe la liceo. Sed li malsukcesis la jarfinan ekzamenon kaj devis ĝin reprovi nun.

Karlo rimarkis, ke la plimulto el la knaboj havas sian ĉapon flanke aŭ malantaŭe sur la kapo. Kredante, ke tio estas kvazaŭ oficiala kutimo, Karlo atendis momenton, kiam neniu lin rigardas, kaj rapide maldekstren puŝis sian ĉapon. Sed tio ne longe utilis ĉar la sonorilo tuj eksonoris, kaj ĉiuj knaboj ekiris al la ŝtuparo super la kolonoj. Sekvante unu la alian kvazaŭ ŝafoj, ili supreniris. Supre ili trovis profesoron kun longaj lipharoj, kiu ordonis, ke ili malsupreniru kaj eniru la trian pordon ĉe la maldekstra konstruaĵo.

Apud la malfermita pordo troviĝis la direktoro parolanta kun tre altkreska profesoro. Tiu ĉi senpense saltigadis la ŝlosilon de la klaso super sia mano kaj ĝin ĉiufoje rekaptis. Malantaŭ Karlo eniris malgrasa kaj pala knabeto kun bone kombitaj haroj. Li ŝajnis tre bonmaniera kaj ankaŭ timema. Jam en la korto li forprenis sian ĉapelon por peti sciigon de Belnett, kies laŭta ekridego forflugigis lin kvazaŭ timigatan birdeton.

Kiam ĉiuj estis en la klaso, la malgrasa knabeto venis sidiĝi apud Karlon. La granda profesoro envenis kaj energie fermis la pordon. Karlo opiniis, ke li estas certe bona kaj afabla viro, ĉar li havas tiel brilajn okulojn kaj tiel belan nigran barbon.

— Mi diktos nun la ekzamenajn demandojn, li diris; vi havos unu skribverkaĵon, kiu utilos samtempe kiel ortografia, historia kaj geografia ekzameno. Skribu: Kion vi scias pri Afriko? Vi devas skribi almenaŭ tri paĝojn kaj ne pli ol kvar. Nun la dua demando estas problemo; skribu: Iu Sinjoro A aŭtomobile veturas kun rapideco de dudek-sep mejloj en unu horo, kaj alia aŭtomobilisto B veturas sur la sama strato kun rapideco de kvardek-unu mejloj en unu horo; kiam la dua renkontos la unuan, supozite ke S-ro A forveturis de la urbo je la naŭa kaj duono matene kaj S-ro B forveturis de la sama loko je la deka matene?

Poste komencis la silenta laborado. De tempo al tempo iu brue ektusis. La granda profesoro promenadis tra la ĉambro. Ĉiufoje kiam li apudpasis, la pala knabeto flanke de Karlo ektremis kaj haltis en sia skribado.

\newpage

\section*{Demandoj}

\begin{enumerate}
    \item  Kiun renkontis Karlo ĉe la ekzameno?
    \item  Kial Belnett devis provi la ekzamenon?
    \item  Kion rimarkis Karlo?
    \item  Kion faris la profesoro, kiu parolis kun la direktoro?
    \item  Kion diktis la profesoro?
    \item  Ĉu la problemo estis tre malfacila?
    \item  Ĉu vi jam vidis kuradon de aŭtomobiloj?
\end{enumerate}
