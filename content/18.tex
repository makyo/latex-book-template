\chapter{Alico}

Ĉiusomere Karlo veturis hejmen por pasigi kelkajn semajnojn kun siaj gepatroj kaj Heleno. Post sia dua jaro en Bruselo, kiam li revenis hejmen, li trovis sian fratinon fianĉino de juna advokato en la urbo. Lia estonta bofrato estis tre bone konata pro sia delikata elokventeco kaj ankaŭ pro sia afablega karaktero, kiu havigis al li multajn amikojn. Tiu fianĉiĝo kaŭzis multe da invitoj de kaj al la familio Davis.

Tiun someron Karlo vidis kaj revidis multajn personojn en la urbo. La fianĉo de Heleno iam venis peti Karlon, ke li akompanu sian fratinon kun li por viziti amikon de lia patro, kiu ĵus revenis el Hindujo. Estis profesoro Palam, de la Kalkuta Universitato. Li kun sia familio revenis el Kalkuta por pasigi en la urbo du jarojn da libertempo, kiujn li postulis por sia sano.

Dum la juna advokato paroladis pri la talento de la profesoro, homo tiel interesa, tiel klera, k.t.p., Karlo sentis malnovan fajron rebruliĝantan en sia koro.

Pri la junulino ekvidita en la preĝejo li repensis. Pri la perdita pilko en la parko li rememoris, ankaŭ pri la promenadoj ĉirkaŭ la mistera domo. ``Mi plezure vin akompanos tien,'' li diris.

La tagon, kiam, kune kun la gefianĉoj, li iris al la domo de profesoro Palam, Karlo estis neordinare ekscitita. Heleno tre miris pri tio.

La profesoro kun sia edzino akceptis ilin ĉarme kaj prezentis al Karlo sian filinon. La junulino estis plej bela kaj dolĉa, nun kredeble dudek aŭ dudek-unujara.

``Sinjoro Karlo Davis,'' diris la profesoro, ``mia filino Alico.'' Kvankam tre konfuzita, Karlo fariĝis kuraĝa. — ``La grandan plezuron renkonti vin, fraŭlino, mi jam havis antaŭ kvin jaroj en la Nacia Parko, kie vi perdis pilkon.''

Ŝi ĉarme ridetis iom ruĝiĝante: ``Ho, jes, sinjoro, mi memoras, ŝi diris, vi estis tiel afabla!''

Plej agrablan vesperon pasigis Karlo ĉe la hejmo de profesoro Palam. Tien li ofte revenis. Oni petis, ke li legu versaĵojn siajn. Eĉ kanton li verkis, kiun lernis fraŭlino Alico por kanti laŭ ario ŝatata de ŝi.

Ofte fraŭlino Palam estis invitita de fraŭlino Davis; kaj Karlo pro tio estis al sia fratino duoble pli afabla ol antaŭe. Okazis iam, ke Karlo iris kun Alico remadi sur la rivero. Li luis boaton kaj ambaŭ trankvile remis ĝis la arbareto de salikoj. Kiam por vespermanĝo ili revenis hejmen, gefianĉoj ili estis.

La konsenton de gesinjoroj Palam, Karlo facile akiris, same kiel la aprobon de siaj gepatroj. Li estis la plej feliĉa junulo en la mondo.

\begin{quotation}
    Faru demandojn por la jenaj respondoj.
\end{quotation}

\newpage

\section*{Demandoj}

\begin{enumerate}
    \item  — ? Jes, ĉiusomere.
    \item  — ? Kun siaj gepatroj kaj Heleno.
    \item  — ? Kun juna advokato en la urbo.
    \item  — ? Jes, li havis multajn amikojn.
    \item  — ? Ĉar li havis afablan karakteron.
    \item  — ? Li estis profesoro en la Kalkuta Universitato.
    \item  — ? Li revenis por sia sano.
    \item  — ? Li diris al Karlo, ke la profesoro estas tre klera.
    \item  — ? Jes, li estis akceptata tre afable.
    \item  — ? Ŝi estis ĉirkaŭe dudekjara.
    \item  — ? Ĝi estis verkita de Karlo.
    \item  — ? Jes, li estis tre feliĉa.
\end{enumerate}
