\chapter{Ĉagrenoj}

Post kelkaj tagoj Janko povis doni al sia amiko la sekvantajn informojn pri la mistera domo: ĝi apartenas al profesoro de la Kalkuta Universitato. De dudek jaroj li vivadas en Hindujo kun sia familio; kaj nur trifoje li jam revenis pasigi kelkajn monatojn en la urbo. La junulino estas sendube lia filino, ĉar oni scias, ke li havas nur unu infanon. Tiuj sciigoj ne estis tute ĝojigaj por Karlo.

Reveninte hejmen unu tagon, li trovis sian fratinon ploranta: la avo mortis. Gesinjoroj Davis jam foriris al lia domo kun onklo Jako, kiu venis ilin sciigi. Karlo tuj remetis sian ĉapelon, kuris al tramveturilo kaj baldaŭ alvenis en la avan domon. Suprenirinte la ŝtuparon, li kvazaŭ ne plu povis spiri. Ĉiaj ideoj kaj sentoj miksiĝis en lia cerbo. Li miris, ĉu li bone komprenis, kion Heleno diris. Ĉu vere li neniam plu vidos la bonan avon, tiel bonkoran? Lia patro malfermis la pordon kaj premante sian filon al sia brusto, lin kisis multfoje. Karlo ekploris. Lia patrino kondukis lin al la ĉambro de l'mortinto.

Tie sur lito kuŝis la avo, kun vizaĝo tute blanka. Li ŝajnis dormanta. Estis nenia rideto sur liaj lipoj, sed tamen aspekto kvieta kaj fida. Longe restadis Karlo apud la korpo de sia avo. Li repensis pri la tempo, kiam li aŭskultis rakontojn sur liaj genuoj; pri la gajaj vesperoj kun li pasigitaj; pri liaj ŝercoj kaj vojaĝaj aventuroj; pri lia bonkoreco al ĉiuj.

Iam al Karlo la avo diris: En ĉiu homo estas anĝelo kaj diablo. Tre ofte unu el ambaŭ estas multe pli potenca, ol la alia. Eĉ kiam ĝi estas la diablo ĉiam serĉu la anĝelon. Forgesu la difektojn de l'aliaj, memoru iliajn bonajn ecojn. Ĉiam parolu al ilia bona parto, neniam al la malbona. Tiel vi kuraĝigos multajn homojn kaj vin mem feliĉigos. Vi plifortigos la bonajn ecojn kaj ekdetruos la malbonajn. La avo tiel bone sciis trovi la bonajn ecojn de ĉiu!

La tagon de l'funebro la domo de l'avo estis plena de floraj bukedoj kaj kronoj donacitaj de amikoj de la familio. Vizitantoj alvenadis ĉiumomente, redirantaj ĉiam la samajn kutimajn frazojn. Kiel ĉio ĉi estis dolora, teda, dum Karlo volus esti sola, trankvila kun siaj gepatroj!

Post la diservo en la preĝejo, ĉiuj iris al la tombejo, sekvante la funebran ĉerkoveturilon. La tombejo estis malseka pro ĵusa pluvo. Ĉiuj kunvenis ĉirkaŭ senherba loko. Estis granda fosaĵo en la tero kaj planko ambaŭflanke. Oni glitigis la ĉerkon malsupren kaj ĝin tuj rekovris per tero. La pastro ekpreĝis, sed Karlo pripensis malĝoje. En liaj oreloj ankoraŭ sonis la bruo de la tero falanta sur la ĉerkon. Tien oni metis lian karan avon.

Jam la personoj disiĝis por reiri hejmen. Pasante inter la aliaj tomboj, Karlo malĝoje ekpensis pri ĉiuj, kiuj jam kuŝas tie sub la tero.

\newpage

\section*{Demandoj}

\begin{enumerate}
    \item  Ĉu Janko povis — al sia — informojn? Jes, — povis, post kelkaj —.
    \item  — mortis en la — de Karlo? Lia —.
    \item  Kie — la avo? — — lito.
    \item  — oni kondukis la — ? Al la tombejo.
    \item  Per — oni rekovris la — ? Oni rekovris — per tero.
    \item  — kio pensis Karlo? — pensis pri la —, kiuj jam kuŝas sub la —.
\end{enumerate}
