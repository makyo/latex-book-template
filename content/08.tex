\chapter{La liceo}

Kiam Karlo estis dekdujara, liaj gepatroj opiniis, ke li sufiĉe longe restis en infana lernejo. Estis tempo meti lin en liceon. (Liceo estis, en la tempo de nia rakonto, la nomo de la oficialaj lernejoj, kie la knaboj lernadas ĝis ili fariĝas studentoj). Estis unu liceo en la urbo. Bedaŭrinde ĝi estis malproksime kaj Karlo devos marŝadi dudek minutojn por tien iri de sia hejmo. Lia patrino proponis, ke oni aĉetu por Karlo biletaron por la tramvojo, sed S-ro Davis tute ne permesis tion. Li opiniis, ke marŝado estas tre bona kaj saniga ekzercado.

Por esti akceptata en la liceon, estis necese ke Karlo sukcesu je ekzameno antaŭe. La ekzameno estis nek longa, nek malfacila. Tamen S-ro Davis volis certiĝi, ke lia filo sukcesos; por tio li venigis hejmen junan instruiston, kiu en kelkaj semajnoj pliprogresigis Karlon en la kono de aritmetiko kaj ortografio, ol Sinjorinoj Linar en kelkaj jaroj.

Unu semajnon antaŭ la ekzameno, S-ro Davis kondukis sian filon al la liceestro por lin enskribigi. La estro estis tre malalta viro kun grizaj haroj, griza barbeto kaj rondaj, brunaj okuloj post oraj okulvitroj.

— Sinjoro direktoro, diris S-ro Davis, kiam la servisto lin enirigis kun lia filo en la direktejon, kiel vi estas? Mi ĝojas vin revidi, kaj alkondukas al vi mian knabeton Karlon, kiu estos espereble bona lernanto en la liceo.

La direktoro rigardis Karlon tra siaj grandaj okulvitroj kaj malfermis larĝan libregon.

— Davis, Karlo\ldots{}kiu estas via alia antaŭnomo, junuleto?

— Teodoro, Sinjoro direktoro, respondis Sinjoro Davis vidante, ke lia filo distrate ne aŭdis la demandon. Efektive Karlo estis distrata: li ĵus vidis muŝon, kiu dronis en la kupran inkujon kaj li atente observis ĝiajn penojn por forflugi.

— Via aĝo? demandis la direktoro.

S-ro Davis pinĉis la brakon de Karlo: ``Kiom jara vi estas, Karlo?''

Karlo ektremis kaj serioziĝante, lia vizaĝo tute ruĝiĝis: ``Dekdujara, sinjoro,'' li diris timeme.

— ``Vi venu lundon," diris la direktoro, "je la 8-a matene kun inko, plumo, krajono, kaj blanka papero por la ekzameno. Via filo, Sinjoro Davis, ŝajnas bona kaj inteligenta knabeto. Bonan tagon\ldots{}atentu la ŝtupon post la dua pordo maldekstre.''

Elirante el la skriboĉambro de l'direktoro, S-ro Davis montris al sia filo la internan korton de la liceo. Meze de la du novaj konstruaĵoj staris la malnova parto de la lernejo, kun duobla ŝtuparo super kolonoj kaj arkaĵoj.

— Tien venadis multaj generacioj antaŭ mi kaj vi, diris al Karlo lia patreto. Via avo kaj niaj praavoj tie lernadis, kiam ili estis knaboj.

\newpage

\section*{Demandoj}

\begin{enumerate}
    \item  Kiun aĝon havis Karlo, kiam li ekiris al liceo?
    \item  Ĉu la liceo estis proksime de lia hejmo?
    \item  Ĉu Karlo estis multe lerninta ĉe S-inoj Linar?
    \item  Kion diris la patro al la direktoro?
    \item  Pri kio okupiĝis Karlo, kiam oni demandis lin?
    \item  Kion li devis alporti por la ekzameno?
    \item  Kion montris S-ro Davis al sia filo?
    \item  Ĉu la lernejo estis tute nova?
    \item  Ĉu vi jam provis ekzamenon?
\end{enumerate}
