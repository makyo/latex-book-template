\chapter{Dimanĉo}

Dimanĉe matene Karlo restis longe en sia liteto kaj ĝuis la dolĉajn sonĝojn, kiujn alportas la oraj sunradioj, enirante en la ĉambron inter la kurtenoj de l'fenestro\ldots{}Subite li ekvidis la dolĉan vizaĝon de sia patrino, alveninta por lin veki. Li estis tre gaja. Laŭte sonoradis la sonoriloj de l'preĝejoj, dum lia patrino lin helpis por lia tualeto.

Karlo opiniis, ke la spongoj estas tre malagrablaj objektoj, kiuj metas sapon en liajn okulojn. Li jam ofte konigis tiun impreson al sia patrino, sed ŝi tamen daŭrigis la uzadon de spongoj. Dimanĉe S-ino Davis vestis Karlon per brodita flanelĉemizeto, blanka pantalono kaj blua jaketo. Li havis flavajn ŝuetojn kaj grandan pajlan ĉapelon, pri kiu li estis tre fiera.

Dimanĉe posttagmeze Karlo promenadis kun sia patro. Li ŝatis iri al la bordo de l'rivero por vidi la ŝipojn kaj la fiŝkaptistojn, kiuj estis tre multaj. Estis seriozaj sinjoroj, longbarbaj maljunuloj kaj ankaŭ multaj knaboj. Ĉiu silente sidis sub unu el la arboj apud la bordo. Ili pacience rigardadis la fadenon de sia fiŝkaptilo malaperantan en la akvo. De tempo al tempo iu levis subite sian kanon kaj dekroĉis de la hoko brilantan fiŝon, malespere saltantan ĉiuflanken.

Antaŭ la vespero S-ro Davis kaj lia fileto revenis hejmen, kaj alvenis ĝustatempe por trinki varmegan tason da teo en la salono.

Ofte la avo kaj onklo Jako vizitis ilin dimanĉe kaj ĉiuj vespermanĝis kune.

La onklo Jako kelkafoje alportis sian fonografilon, kaj Karlo aŭdis mirindajn kantojn kaj muzikaĵojn. Iam onklo Jako diris ŝerckanteton antaŭ la granda buŝo de l'fonografilo. La maŝino ripetis la kanton poste kaj eĉ la ridegon de l'avo kaj de l'patro. Onklo Jako klarigis, ke la fonografilo ripetus kion ajn Karlo dirus antaŭ la maŝino, sed Karlo ne kuraĝis ion diri. Li jam opiniis, ke ofte estas tre saĝe silenti.

Karlo tre ŝatis sidi sur la genuoj de sia avo kaj aŭdi rakontojn pri la infaneco de sia patreto aŭ pri malproksimaj landoj, kiujn la avo vizitis. Li ŝatis aŭdi pri la maro, kaj tre deziris ĝin iam vidi.

Kiam estis bela vetero, la tuta familio sidis en la ĝardeneto. La avo prenis sian pipon, ĝin plenigis per tabako kaj ĝin ekbruligis. La patro kaj la onklo fumis nur cigaredojn. Karlo tre malŝatis la tabakfumon, kiu doloris liajn okulojn. Li ofte demandis sian patron, kial la viroj fumas; kaj la patro ĉiam respondis: ``por la plezuro''. Karlo decidis, ke li trovos pli agrablan plezuron, ol fumadi, kiam li estos viro. Sed kiam lia avo diris al li, ke ĉiuj maristoj fumas, li ekŝanĝis sian opinion.

\newpage

\section*{Demandoj}

\begin{enumerate}
    \item  Kion faris Karlo dimanĉe matene?
    \item  Kiu helpis lin por lia tualeto?
    \item  Kion li opiniis pri la spongoj?
    \item  Per kio vestis sian filon S-ino Davis, dimanĉe?
    \item  Kiuj personoj estis sur la bordo de la rivero?
    \item  Ĉu ili faris multe da bruo?
    \item  Kiu venis ofte por vespermanĝi?
    \item  Kion alportis la onklo?
    \item  Pri kio parolis la avo?
    \item  Kie sidis la familio dum somero?
    \item  Nomu la kvar sezonojn de la jaro.
    \item  Kiun vi preferas?
    \item  Ĉu la patro fumis?
    \item  Kiom kostas bona cigaro?
    \item  Ĉu la maristoj ofte fumas?
\end{enumerate}
