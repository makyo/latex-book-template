\chapter{La familio}

Kiam naskiĝis Karlo, rozkolora kaj sana, liaj gepatroj estis ankoraŭ tre junaj. Ilia unua infano li estis. Lia apero en tiu ĉi mondo kaŭzis grandegan ĝojon en la tuta familio. Filo! vera vivanta knabeto! La junaj gesinjoroj Davis estis tre fieraj pri Karlo kaj ĝuis internan kaj trankvilan plezuron.

Sed la patro de S-ro Davis estis tro ĝoja por esti trankvila. Avo li estis nun; li tute forgesis, ke tio lin iom maljunigas kaj li kontraŭe sentis en si infanan gajecon.

Dum du tagoj post la naskiĝo de Karlo, li ne povis labori en sia oficejo. Li kuradis de sia domo al la domo de sia filo kaj de tie revenis al sia domo; li promenadis tra la stratoj de l'urbeto, eniris en ĉiajn magazenojn por aĉeti aferojn tute senutilajn; al ĉiu almozpetanto li donis multajn monerojn. Oni povis facile vidi, ke li estas kontenta.

La gepatroj de Sinjorino Davis estis jam tre maljunaj. Ili loĝis en granda urbo, kaj oni telegrafe anoncis al ili la naskiĝon de ilia nepo. Por ili, li ne estis la unua. Ilia filino Margareto, la feliĉa patrino de Karlo, estis ilia sepa infano. Jam du el ŝiaj fratoj kaj tri el ŝiaj fratinoj estis edziĝintaj kaj eĉ havis infanojn, kiam fraŭlino Margareto renkontis la junan S-ron Davis, kun kiu ŝi baldaŭ fianĉiniĝis kaj poste edziniĝis. Fraŭlino Margareto estis tre beleta junulino kaj tre dolĉa. Pro tio ĉiuj gefratoj tre amis ŝin kaj ŝiaj gepatroj ricevis grandan ĉagrenon, kiam ŝi devis forveturi kun sia juna edzo.

Ĉe la baptotago de Karlo, preskaŭ la tuta familio de Sinjorino Davis ĉeestis, kaj ankaŭ la patro kaj la juna frato de Sinjoro Davis. Oni nomis la knabon Karlo-Teodoro por kontentigi lian patran avon S-ron Karlo Davis kaj lian patrinan avon S-ron Teodoro Renberg. Kompreneble oni ne povis elekti alian baptopatron, ol lian patran avon. La baptopatrino estis la plej maljuna fratino de Sinjorino Davis.

Dum la pastro malsekigis lian beletan kapeton kun kreskantaj blondaj haroj, sinjoreto Karlo-Teodoro ŝajnis gaje rideti kaj, subite palpante siajn malsekajn harojn per siaj manetoj, li faris al la ĉeestantoj en la preĝejo laŭtan, sed nekompreneblan paroladon, kiu preskaŭ rompis la seriozecon de la ceremonio. Sendube por haltigi lin, lia onklino donis al li pluvegon da kisoj. Kredante, ke ili estas gratulesprimoj kaj estante jam tre modesta, li tre malafable akceptis tiujn kisojn. Jam tie li mirigis multajn personojn.

\newpage

\section*{Demandoj}

\textbf{Grava rimarko}

\noindent Por respondi al la demandoj oni devas neniam uzi simple ``jes'' aŭ ``ne'', sed ordigi la vortojn de la demando mem alimaniere kaj aldoni la necesajn vortojn.

\emph{Ekzemple}: Ĉu la gepatroj de Karlo estis junaj, kiam li naskiĝis? — Jes, la gepatroj de Karlo, Sinjoro kaj Sinjorino Davis, estis ankoraŭ junaj, kiam naskiĝis ilia filo.

\begin{enumerate}
    \item Ĉu la gepatroj de Karlo estis junaj, kiam li naskiĝis?
    \item Ĉu Karlo estis la unua infano de S-ro Davis?
    \item Kion faris la patro de S-ro Davis?
    \item Kiel oni anoncis la naskiĝon al la gepatroj de S-ino Davis?
    \item Kiom kostas la sendo de telegramo?
    \item Donu kelkajn baptonomojn de knaboj kaj knabinoj.
    \item Kion oni faris al Karlo en la preĝejo?
\end{enumerate}
