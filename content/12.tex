\chapter{La amiko}

Tre feliĉa estis Karlo, ĉar li havis veran amikon. Lia nomo estis Janko. Li estis tre altkreska kun nigraj haroj kaj brunaj okuloj. Nur ses monatojn pli aĝa ol Karlo li estis, tamen ĉiuj respektis lin en la klaso pro lia kuraĝo. Efektive nenion li timis kaj li estis plej lojala kaj malkaŝema knabo.

Iam la profesoro de geografio, vidante, ke li oscedas ĉiumomente, lin demandis: ``Janko, ĉu vi enuas tie ĉi?'' --- ``Jes, Sinjoro, tre multe,'' li respondis. Forpelita li estis, sed nur por unu horo.

Janko bonege desegnis kaj ankaŭ pentris. Li havis plej mirindan talenton por prezenti tute precize ies fizionomion per kelkaj krajonaj strekoj, kaj li desegnis en sia kajero la vizaĝon de ĉiuj siaj amikoj. Ke preskaŭ ĉiuj profesoroj ŝatis Jankon, estis facile rimarkebla, kvankam ili ofte ŝajnis lin timi.

Karlo lin samtempe amis kaj respektis. Li ĉiam klopodis esti apud li kaj lin sekvadi. Li tre bedaŭris, ke li ne povas sidi en la klaso apud Janko, kvankam Man estis afabla najbaro. Karlo tre ofte alvenis hejmen malfrue post la fino de la posttagmezaj lecionoj. Kiam lia patrino demandis lin: ``Nu, Karlo mia, kion vi faris tiel longe?'' --- ``Mi akompanis Jankon,'' estis lia ĉiama respondo.

Janko sciis tiom da aferoj! Li estis vera scienculo, kaj Karlo opiniis, ke li jam instruis lin multe pli ol la lecionoj en la liceo. Tamen Karlo ĉiam sentis sin nesciulo apud li, sed Janko neniam mokis lin pri eraro aŭ nescio. Karlo opiniis, ke tio estas pruvo de senfina delikateco en la karaktero de lia amiko.

Estis vera honoro tiel esti protektata de Janko, kaj fiere Karlo ĝuis ĝin. Al Janko li diris ĉiujn sekretojn siajn kaj petis konsilojn en ĉiu embarasanta okazo. Kiam Karlo faris al li demandon pri malfacila afero, li kelkafoje diris: ``Mi respondos al vi morgaŭ.'' La morgaŭan tagon, sen ia forgeso, li alportis plenan informon pri la afero.

Lia patro estis presisto kaj havis grandegan librejon. Kelkafoje Janko pruntis librojn al Karlo kaj eĉ al aliaj knaboj, sed sur la unuan paĝon li ĉiam metis antaŭe per kaŭĉuka stampilo sian nomon kaj la jenajn versojn de fama poeto:

\begin{quotation}
    \noindent ``De pruntita libro jen la nepra sorto:\\
    Ofte ĝi perdiĝas, ĉiam difektiĝas.''
\end{quotation}

Pro sia kontraŭema karaktero, la knaboj preskaŭ ĉiam redonis la librojn en sufiĉe bona stato.

En somero, dum la libertempo, Janko kaj Karlo faris multajn ekskursojn, ĉu piede, ĉu biciklete, kaj ĉiam pligrandiĝis ilia amikeco.

\begin{quotation}
    \noindent Anstataŭi la streketojn per taŭgaj vortoj.
\end{quotation}

\newpage

\section*{Demandoj}

\begin{enumerate}
    \item  Ĉu Karlo — feliĉa? Jes, li estis — ĉar li havis veran —.
    \item  — Kiu estis la — de — amiko? Lia — estis Janko.
    \item  — Ĉu li estis multe — aĝa ol Karlo? Ne, li nur — ses — pli aĝa.
    \item  Kion demandis la — de geografio? — demandis: Janko, — vi enuas — ĉi?
    \item  Ĉu Karlo estis — de Janko? Ne, — ne — najbaro de —.
    \item  Ĉu Janko — sciis? Jes, Janko estis — scienculo.
    \item  Kion — Karlo al Janko? Li diris siajn —.
    \item  Ĉu Janko ĉiam tuj — la demandojn? Ne, — li respondis la morgaŭan —.
    \item  Kion faris la — de Janko? Li estis —.
    \item  Kiu viro — la presarton? Gutenberg elpensis la —.
    \item — publika biblioteko — via urbo? Jes, en — urbo estas unu — biblioteko.
\end{enumerate}
