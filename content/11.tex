\chapter{Latina leciano}

La profesoro de latina lingvo estis klariganta la duan deklinacion de la substantivoj. Li havis longan barbon, kiu iam estis tute blanka sed kiu flaviĝis iom post iom. Liaj bluaj okuloj montris bonkorecon.

— Lupus, lupi, li diris per sia laŭta kantema voĉo, discipulus, discipuli, jen du substantivoj de la dua deklinacio, la lupo kaj la lernanto. Ili finiĝas nominative per -us, dum la substantivoj de la unua\ldots{}Rigar! vi dormas, ĉu ne? stariĝu! Kiel finiĝas la substantivoj de la unua deklinacio?

Rigar stariĝis silente kaj ekrigardis la profesoron.

— Vi skribu tion ĉi kvindekfoje por morgaŭ: nauta, nautae; insula, insulae, en ĉiuj kazoj!

Rigar residiĝis gratante sian orelon.

— Discipulus, discipuli, discipulum\ldots{}kante klarigis la instruanto. Karlo pripensadis dormeme. Estis tiel varme. Je kio utilas la lernado de la latina lingvo? li pensis. Neniu ĝin parolas nun, lia patro diris. Tamen pri ĝi estis io mistera, io antikva, kiu plaĉis al Karlo. En la latina lingvo estas ĉiuj malnovaj surskribaĵoj, sur la muroj kaj en la preĝejoj. La sinjorinoj ne komprenas la latinan lingvon, nek multaj aliaj personoj. La preĝoj kaj la meso en la katolikaj preĝejoj estas latine dirataj.

Dum Karlo pripensis, lia najbaro Man pacience laboradis. Per sia poŝtranĉilo li jam engravuris sur la benko sian tutan antaŭnomon kaj nun komencis grandan M. Sed bedaŭrinde lia tranĉileto estis tro delikata kaj subite ĝi brue rompiĝis.

La profesoro malrapide alpaŝis.

— ``Man, knabo mia, mi devos vin severe puni. Kion vi skribis? Vian nomon? Malsprita amuzaĵo! Morgaŭ vi ne scios ion pri la dua deklinacio. Tiuj benkoj estas en terura stato, ĉiuj estas gravuritaj; vi ne havas iom da respekto al la licea meblaro.'' Li ekpromenis tra la klaso. ``Stanen, ankaŭ vi skribis vian nomon kaj ĝin per makuloj ĉirkaŭis! Delasar, vi desegnis tiun ĉi vizaĝon, ĉu ne?''

Sur ĉiuj benkoj estis efektive desegnaĵoj kaj gravuraĵoj diversaj. Estis tre antikva kutimo de la liceanaj, engravuri sian nomon por lasi memoraĵon al la posteuloj.

``Mi opinias, ke mi devas vin ĉiujn puni,'' diris la profesoro, daŭrigante sian promenadon inter la benkoj. ``Kvankam tiuj tabuloj estas tre malnovaj, vi estas tamen ĉiuj kulpaj, tre kulpaj. Nur bubetoj kaj malsaĝuloj tiel amuziĝas anstataŭ\ldots{}'' Subite li haltis paliĝante antaŭ la benko de Gardiol\ldots{}Dum longa momento li silentis. Lia vizaĝo estis blanka. Liaj okuloj ŝajnis ligitaj al la tabulo de tiu benko. Neniu en la klaso kuraĝis moviĝi aŭ eĉ spiri.

``Knaboj, li ekdiris per mallaŭta kaj tremanta voĉo, tie estas gravurita la subskribo de\ldots{}mia patro.''

De tiu tago, Karlo havis grandan amon al la profesoro de latina lingvo.

\newpage

\section*{Demandoj}

\begin{enumerate}
    \item  Kion diris la profesoro de latina lingvo?
    \item  Ĉu Rigar aŭskultis?
    \item  Ĉu vi parolas angle?
    \item  Kiom de tempo vi lernis Esperanton?
    \item  Kion faris Man?
    \item  Ĉu la profesoro lin gratulis?
    \item  Ĉu la lernantoj sidis sur seĝoj?
    \item  Kion ekvidis la profesoro sur unu tablo?
    \item  El kiuj lignoj oni faras tablojn?
    \item  Nomu la kolorojn, kiujn vi konas.
\end{enumerate}
