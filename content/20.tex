\chapter{En Italio}

Ĉar Karlo estis nun tre okupata kuracisto, li ne povis resti longe for de la urbo. Pro tio la edziĝovojaĝo estis ne tre longa. Aliflanke Alico forte deziris baldaŭ komenci sian edzinan vivadon en la nova hejmeto. Sed ŝi ankaŭ deziris viziti Italion\footnote{En la originalo «Italujo», kiu estas arkaika formo.}, kies nur unu havenon ŝi konis.

La junaj geedzoj traveturis Svislandon, haltante en la bela urbo Lucerno kaj celante Milanon per la Gotharda fervojo. La tuta pejzaĝo de tiu vojo inter la montoj tre forte impresis ilin per sia grandioza beleco.

Alveninte Luganon, ili estis tiel ĉarmitaj de la lago, ke ili denove haltis kelkajn tagojn kaj ankaŭ vizitis Komon kaj la ĉirkaŭan regionon. Akvoj bluaj, bluegaj, en kiuj fandiĝas oraj sunradioj, krutaj montoj silente gardstarantaj ĉirkaŭe, blankaj domoj, ĝardenoj kun vinbero kreskanta sur arboj kaj muroj: tio estas la lando de l'poezia trankvileco, kie Plinus revante promenadis antaŭ dumil jaroj.

En Milano ili pasigis tutan matenon en la turoj kaj sur la tegmento de l'grandega ĉefpreĝejo, kiu blanke brilegis kvazaŭ giganta juvelo sub la hela ĉielo.

En Venezio ili longe promenadis en la palaco de l'dukoj, admirante la grandajn majstroverkojn de l'pentristoj el la venezia skolo. En nigra gondolo ili dolĉe glitadis sur la kanaloj inter la marmoraj palacoj kaj sub la arkaj pontoj. Ili ŝipveturis al la insulo Lido por vidi la maron Adriatikan. Antaŭ la ora ĉefpreĝejo de Sankta Marko ili disĵetis grenerojn al la kolomboj sur la Granda Placo. Vespere, sur la Granda Kanalo, inter la kolorpaperaj lanternoj ili sekvadis en gondolo koncertboaton, el kiu kantistoj kun gitaroj aŭ mandolinoj sonigis tra la nokto melodian arion.

En Bolonjo ili vizitis la malnovajn preĝejojn por admiri la pentraĵojn. En Romo ili vidis la Forumon, kiun ĉirkaŭas tiom da gloraj ruinaĵoj kaj plenigas tiom da eternaj memoraĵoj.

En Napolio ili vagadis ĉe l'marbordo kaj vidis nepriskribeblajn subirojn de la suno. La vaporŝipego, en kiu vojaĝis gesinjoroj Palam por reiri al Kalkuta, estis haltonta ĉe Napolio. Ĝian alvenon ili atendis, por ilin adiaŭi. Poste ili revenis hejmen haltante ankoraŭ en Nice kaj Marseille.

Tre ĝoje ili komencis sian novan kunvivadon hejme, kaj Karlo feliĉkore miris, kiel bele realiĝis lia knaba revo.

\newpage

\section*{Demandoj}

\emph{Rakontu skribe la supran ĉapitron. }

\begin{enumerate}
    \item  — ? Li estis kuracisto.
    \item  — ? Ne, ĝi ne estis tre longa.
    \item  — ? Ŝi deziris viziti Italujon.
    \item  — ? Jes, tiu lando estas tre bela.
    \item  — ? Ili haltis en Lucerno.
    \item  — ? Jes, tiu urbo estas ĉe la bordo de lago.
    \item  — ? Ili restis en Lugano dum kelkaj tagoj.
    \item  — ? La restoracio en la stacidomoj estas nomata bufedo.
    \item  — ? Estas la lokomotivo, kiu trenas la vagonaron.
    \item  — ? Jes, dum la vintro la svisaj montoj estas tute kovrataj de neĝo.
    \item  — ? Jes, Venezio estas tre malnova urbo.
    \item  — ? Sur la kanaloj oni veturas per gondoloj.
    \item  — ? Jes, tiuj ŝipetoj estas tre graciaj.
    \item  — ? La sono de la violono estas pli bela, ol tiu de la gitaro.
    \item  — ? La vaporŝipo estis haltonta ĉe Napolio.
    \item  — ? Jes, Karlo fariĝis konata kaj bone sukcesis.
\end{enumerate}
