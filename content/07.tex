\chapter{La sonĝo de Karlo}

Post la interparolado kun sia patro pri la formo de l'tero, Karlo pripensis ankoraŭ tre ofte dum la tago. Strangan sonĝon li havis la sekvintan nokton. Li sonĝis, ke S-ro de Lavel, la riĉa amiko de lia patro, lin invitis por kunvojaĝado en sia aeroplano.

Ili aĉetis grandan keston da biskvitoj kaj metinte ĝin sur la aeroplanon, ili sidiĝis kaj ekflugis for. Ili devis transiri altajn montojn. La aeroplano flugadis tre rapide kaj trapasis grizajn nubojn. Estis malseke kaj malvarme. Karlo sin premis kontraŭ sia gvidanto, dum ili pasis super akraj montpintoj kaj teruraj rokoj.

S-ro de Lavel montris al Karlo, kiel la ebenaĵo subite finiĝas en la malproksimo. Tie estas la ekstremaĵo de la mondo, kiun li tiel deziris vidi. La aeroplano baldaŭ alteriĝis. Marŝinte kelkajn paŝojn, Karlo sin trovis ĉe la limo de la mondo. Estis granda kajo kun fera barilo inter limŝtonoj. Tenante la manon de S-ro de Lavel, Karlo kliniĝis por rigardi malsupren: nenio, nur bluo\ldots{}senfine.

Ambaŭ manĝis kelkajn biskvitojn tute malmoligitajn de la malvarmo. ``Nu," diris S-ro de Lavel post momento, "ni nun veturos malsupren sur nia flugmaŝino, ĉu ne?'' Karlo ektimis iom sed nenion diris. Ili forflugis. La aeroplano transpasis super la barilo kaj flugadis tute rekte for de l'tero.

Post kelkaj minutoj, kiam ili estis jam tre malproksimaj de la tero, S-ro de Lavel ŝanĝis la direkton de la maŝino. Ĝi komencis oblikvan flugadon malsupren. Ili reproksimiĝis al la tero, kaj tiam ĝian subaĵon ili vidis.

Pli kaj pli mallumiĝis, dum la aeroplano rapidege flugadis kurbe por restadi proksime de l'tero. Stranga afero: la ĉielo estis malsupre kaj flanke, sed supre kaj ĉe la alia flanko estis la bruna tero.

La aeroplano estis malsuprenirinta tre rapide kaj povis nun flugadi preskaŭ horizontale. Ili ekvidis strangajn arbojn kun longa torda trunko rampanta. Baldaŭ domojn ili ekvidis, verajn domojn pendantajn sub la tero. Ili estis tre multaj kaj ŝajnis esti lignaj dometoj diverskolore pentritaj. Ĉiam pli kaj pli ili alproksimiĝis. Karlo rimarkis, ke la dometoj havas pintan tegmenton kaj ŝajnas pendi je la tero per granda fera ringo ligita al la trunkoj kaj branĉegoj de la rampantaj arboj.

Inter la domoj estis pontoj, sur kiuj aperis multaj homoj ŝajne blanke vestitaj.

Klininte sin por pli bone vidi, Karlo perdis sian ekvilibron kaj falis en la abismon . . .

Li sentis la malvarman aeron siblantan je liaj oreloj, dum li senfine faladis en la profundegaĵo.

— ``Nu, karuleto, estas malfrue, vi devas ellitiĝi!''

Lia patrino lin vekis. Karlo frotis siajn okulojn: ``Patrineto," li diris oscedante, "mi vidis homojn sub la alia flanko de l'mondo.''

\section*{Demandoj}

\begin{enumerate}
    \item  Kiu invitis Karlon por vojaĝi en aeroplano?
    \item  Kion montris S-ro de Lavel al Karlo?
    \item  Ĉu Karlo iom timis?
    \item  Kion faris la aeroplano?
    \item  Kiujn domojn ekvidis Karlo?
    \item  Ĉu ilia tegmento estis ronda?
    \item  Kio estis inter la domoj?
    \item  Kion faris Karlo por pli bone vidi?
    \item  Kio okazis?
    \item  Kiu vekis lin?
    \item  Rakontu sonĝon, kiun vi faris.
\end{enumerate}
