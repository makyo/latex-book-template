\section*{Antaŭparolo de la redaktoro}

\noindent\emph{Kiam naskiĝis Karlo, rozkolora kaj sana\ldots{}}

\vspace{1em}

% \noindent\begin{paracol}{2}
    Ĉi tiu frazo, pli ol \emph{Bildoj kaj demandoj}, pli ol Adamo kaj Sofia, pli ol iuj ajn vortoj el mia Esperanta edukado restas kun mi. \emph{Karlo} estis unua el miaj ŝatataj enkondukoj al la lingvo.
% \switchcolumn
%     This phrase, more than \emph{Bildoj kaj demandoj}, more than Adamo and Sofia, more than any other words from my Esperanto education remains with me. \emph{Karlo} was one of my favorite introductions to the language.
% \end{paracol}

% \noindent\begin{paracol}{2}
    Nun, post multaj jaroj, mi havas la plezuron por eldoni la libron, ke vi eble ĝuos ĝin, kiel mi ĝuis ĝin.
% \switchcolumn
%     Now, after many years, I have the pleasure of publishing the book so that you might enjoy it as I have.
% \end{paracol}

% \noindent\begin{paracol}{2}
Kiam mi eklaboris eldoni ĉi tion verkon, mi pliboniĝis kiel Esperantisto. Do, kiam mi relegis ĝin, mi rimarkis, ke iuj el la rakonto estis malmodernaj. 110 jaroj estas longa tempo por ambaŭ la mondo kaj Esperanto ŝanĝiĝi. Pro tio, mi zorge redaktas ĉi tion eldonon por moderni la lingvo. Ĉiuj redaktoj havas piednotojn.
% \switchcolumn
%     When I thought bring this work into publication once more, I had grown as an Esperantist. On rereading, I noticed that several bits of the work were quite out of date. 110 years ago is quite a while for both the world and Esperanto to change. To that end, this edition of \emph{Karlo} has been lightly edited to modernize the language. All such edits have been noted in the footnotes.
% \end{paracol}

% \newpage
%
% \noindent\begin{paracol}{2}
    Ĉi tiu libro estas organizita de ĉapitroj, kiu rakontas la vivon de Karlo. Ĉiu ĉapitro enkondukas novajn Esperantajn vortojn, ideojn, kaj konceptojn. Post ĉiu ĉapitro estas multaj demandoj pri la teksto, kiun vi eble respondos. Ni donas al vi pli da spacon inter la teksto por permesi al vi verki notojn pri tio, kion vi legas.
% \switchcolumn
%     This book is organized by chapters which tell the story of the life of Karlo. Each chapter introduces new words, ideas, and concepts in Esperanto. After each chapter are several questions about the text which you may answer. We have given you more space within the text to allow you to write notes as you read.
% \end{paracol}

\vspace{1em}

Amuziĝu, kiel mi amuziĝis!

\vspace{1em}

\noindent\emph{Madison Scott-Clary}

\noindent\emph{Seatlo, en Decembro 2019.}

\newpage

\section*{Antaŭparolo de l'verkinto}

La ``Librairie de l'Esperanto'' en Parizo petis min, ke mi verku facilan legolibron por la lernantoj de kursoj aŭ la komencantoj, kiuj ĵus finis sian lernadon en lernolibro. Mi intencis prezenti dudek tekstojn aranĝitajn tiamaniere, ke en ĉiu estos multaj vortoj pri sama temo de l'ĉiutaga vivo. Por pli bone konservi la intereson de la lernanto en dudek ĉapitroj, mi verkis unu rakonton pri la vivo de junulo, de lia infaneco, ĝis lia edziĝo.

Kompreneble nek romano, nek eĉ sprita novelo estas la rakonto, sed nur aro de dudek ĉapitroj pri ĉiutaga vivo, ligitaj inter si per ĝenerala senco kaj unueco de persono. Tamen ion pli ol lernoĉapitrojn eble trovos en tiu ĉi libro tiuj plenaĝaj personoj, por kiuj ekzistas ia poezia ĉarmo en rememoroj pri infaneco kaj juneco eĉ el plej simplaj okazintaĵoj de l'ĉiutaga vivo.

Mi rekomendas al la instruantoj de kursoj, ke, post lego de ĉapitro, ili faru multajn demandojn pri ĝi kaj donu al la lernantoj bonan okazon paroli la lingvon. (La demandoj montritaj en la libro estas nur bazo, kaj la profesoroj devos plimultigi ilin laŭ la tempo disponebla.)

Aŭdado kaj ripetado de esperantaj frazoj estas la plej bezonata afero ĉe la kursoj, ĉar tion la lernanto ne povas facile trovi hejme en sia lernolibro. Estas necese atentigi la lernanton pri la akcento (ĉiam sur la antaŭlasta silabo) kaj ankaŭ pri natura kaj ne trolonga elparolado de la vokaloj.

\vspace{1em}

\noindent\emph{Edmond Privat}

\noindent\emph{Ĝenevo, en Januaro 1909.}
