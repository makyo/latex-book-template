\chapter{Sur la rivero}

La jaroj rapide forpasis kaj Karlo ĵus eniĝis la gimnazion, kiu estas la pli alta parto de la liceo. Preskaŭ la samajn kamaradojn li havis kiel en la unuaj jaroj. Malmultaj forlasis la lernejon kaj nur unu novulo envenis en la klason de Karlo. Janko ĉiam estis lia fidela amiko kaj lin nun ofte hejmen venigis por diskutadi kaj esplori librojn. Jam de unu jaro Man ne plu sidis apud Karlo, sed estis anstataŭita unue de Laminde, due de Servetti kaj fine de Janko mem.

Iun matenon en somero, estis tiel varme en la klaso, ke preskaŭ ĉiuj ekdormis. Janko diris al Karlo: ``Estas neeble restadi tie ĉi, ĉu ni foriru? — Kien? — Ien ajn. Sur la riveron, ekzemple, ni povus lui boaton.'' Karlo kompreneble entuziasmiĝis je la ideo, kvankam li timis iomete.

Inter du lecionoj, kiam ĉiuj aliaj iris momenton babiladi aŭ kuradi en la korto, ambaŭ rapide trapasis la pordegon kun Servetti, kaj tuj aliris la angulon de la plej proksima strato. Neniu estis vidinta ilin. Ŝajnis strange al Karlo promenadi en la urbo matene. De tiom longe li tion ne faris estante ĉiam en la liceo je tiu tempo.

Ili iris al la bordo de l'rivero. Ili baldaŭ alvenis al boatejo. Janko elektis mallarĝan boaton kaj sidiĝinte, ili ekremis norden. Servetti opiniis, ke estus pli amuze remi komence kontraŭ la fluo de l'rivero kaj reveni poste tute ne remante. Lia propono estis tuj akceptata. La vetero estis belega: sur ambaŭ bordoj de l'rivero la kampoj kaj arbaretoj estis tre verdaj kaj tute dezertaj. Nur la birdojn oni aŭdis kaj la regulan bruadon de la remiloj.

Baldaŭ ili pasis tra loko, kie grandaj salikoj klinitaj super la rivero banis siajn branĉetojn en la akvo. La knaboj malrapidigis sian remadon. Estis dolĉa silento. Ili pensis pri siaj kompatindaj kamaradoj, nun lernantaj grekajn verbojn en la klaso.

Karlo estus dezirinta reveni por la lasta leciono je la dekunua, ĉar li tre amis la profesoron de historio. (Tiu estis la altkreska viro kun bela nigra barbo, kiu diktis la ekzamenajn demandojn, kiam knabeto Karlo la unuan fojon sidis sur benko de la liceo). Eltirinte sian poŝhorloĝon, Karlo ekvidis, ke estas jam tro malfrue.

Ili forlasis la remadon, kaj la boato malrapide kaj dolĉe suden iris portate de la riverfluo. Ili ekkantis, kaj ilia voĉo ĝoje sonis inter la arboj. Jam la domoj de la urbo ekaperis inter la branĉoj kaj baldaŭ ili realvenis al la boatejo.

Ĉiuj rapidis hejmen, ekpensante pri tio, kio okazos, kiam ili reiros la liceon posttagmeze. Ĉe angulo de strato Karlo renkontis\ldots{}la profesoron de historio. Li forprenis sian ĉapelon. La profesoro haltante nur diris: ``Bonan tagon, Davis; vi venu al mia domo morgaŭ je la kvara kaj duono; mi havas ion por diri al vi.'' Kaj li forpasis.

\newpage

\section*{Demandoj}

\begin{enumerate}
    \item  Kio estas la gimnazio?
    \item  Kiujn kamaradojn havis Karlo?
    \item  Kion diris Janko al Karlo iun matenon?
    \item  Ĉu Karlo akceptis la proponon?
    \item  Kion faris la tri junuloj?
    \item  Ĉu la pluvo falis?
    \item  Kion eltiris Karlo el sia poŝo?
    \item  Kioma horo estas nun?
    \item  Ĉu vi scias remi kaj naĝi?
    \item  Kion diris la profesoro al Karlo?
\end{enumerate}
