\chapter{Lernado}

Karlo jam fariĝis forta knabeto kaj rapide grandiĝadis; sed li ankoraŭ ne eklernis la literojn de l'alfabeto. Lia patrino decidis, ke ŝi mem instruos lin en la komenco.

Ĉiumatene ŝi sidigis lin apud si kaj malfermis grandan libron kun nigraj signoj. Ŝi klarigis al li, ke per tiuj signoj oni skribas kaj, se li bone lernas ilin, li povos baldaŭ legi belegajn rakontojn en libroj. Karlo opiniis, ke la homoj estas vere mallertaj kaj sensencaj: kial ili ne venus mem diri siajn rakontojn? Estus multe pli simple. Post pripensado li esprimis sian opinion al la patrino.

— Sed, karuleto, ŝi diris, miloj da personoj volas koni la rakontojn. Estus neeble, ke la rakontistoj vizitu ĉiujn. Multaj rakontistoj loĝas tre malproksime, multaj estas mortintaj nun, kaj oni deziras tamen legi nun tion, kion ili skribis.

— Sed kial ili ne parolis en fonografilo, kiel onklo Jako? Oni ĉiam povus aŭdi ilin.

La patrino trovis Karlon iom tro diskutema:

— Nu, karega, ŝi diris, kisante lin, vi devas eklabori.

Sed Karlo profundiĝis en sia pensado\ldots{}la strangaj nigraj signoj dancadis en nebulo antaŭ liaj okuloj. Li ripetis, A, B, C, senatente. Kiam lia patrino petis lin montri B, li ne sciis. La literon S li bone rekonis, ĉar ĝi similas serpenton, kaj ankaŭ O kaj C, kiuj similas tutan kaj duonan lunon. Por instrui al li la aliajn literojn, la patrino bezonis multe da semajnoj kaj multe da pacienco. Fine gesinjoroj Davis decidis, ke Karlo iru al lernejo. Ne malproksime de la hejmo loĝis maljuna sinjorino kaj ŝia filino, kiuj ambaŭ instruis infanojn. La lecionoj okazis nur tri matenojn ĉiusemajne, lunde, merkrede kaj vendrede.

Sinjorino Davis mem kondukis Karlon al la lernejo la unuan fojon. Dum ŝi parolis kun la respektinda estrino, S-ino Linar, Karlo rimarkigis al Fraŭlino Linar, ke unu seĝo en la klasĉambro estas rompita kaj ankaŭ, ke estas desegnita vizaĝo sur la muro. Li demandis la fraŭlinon, ĉu tio estas ŝia portreto aŭ tiu de la maljuna virino. La fraŭlino treege ruĝiĝis.

— Mi esperas, ke vi estos bona kaj ĝentila, diris lia patrino adiaŭkisante lin.

— Ho jes, patrineto.

Kiam la unua leciono estis komencita, Karlo havis tre baldaŭ siajn manojn tute makulitaj je inko. Laŭ sia kutimo li viŝis ilin per sia antaŭtuketo. Beleta knabineto, kiu sidis apud li, multe ridis kaj moke rigardis lin. Pro tio Karlo opiniis, ke li certe pli amas sian fratinon Helenjon, ol ŝin.

\newpage

\section*{Demandoj}

\emph{Rakontu skribe la supran ĉapitron.}

\begin{enumerate}
    \item  Kiu unue instruis Karlon?
    \item  Kion faris la patrino de Karlo?
    \item  Kion opiniis Karlo?
    \item  Kiujn literojn li facile rekonis?
    \item  Al kio similas la litero S?
    \item  Kiu kondukis Karlon al la lernejo?
    \item  Kion li rimarkis sur la muro?
    \item  Kiun demandon li faris?
    \item  Per kio Karlo makulis siajn manojn?
    \item  Ĉu oni povas skribi sen inko kaj sen plumo?
    \item  Ĉu Karlo estis en geknaba lernejo?
    \item  Per kio li viŝis siajn manojn?
    \item  Kiu ridis pri li?
    \item  Ĉu de multaj jaroj oni uzas la skribmaŝinojn?
    \item  Ĉu vi jam havis okazon uzi skribmaŝinon?
    \item  Ĉu vi estas stenografiisto?
\end{enumerate}
