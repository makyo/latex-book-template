\chapter{Liceano}

Karlo sukcesis la ekzamenon kaj estis akceptita kiel lernanto ĉe la liceo. Matene la lecionoj komenciĝis je la 8-a kaj daŭris ĝis la dekdua, kun dek minutoj da intertempo ĉiuhore inter la lecionoj. Tagmeze Karlo devis rapidi hejmen por ne alveni malfrue por la tagmanĝo. Post la deserto li tuj devis reiri, ĉar jam je la unua kaj duono komenciĝis posttagmeze la lecionoj.

Post kelkaj semajnoj da lernado en la liceo, Karlo jam sciis multe: Li ne kisis plu sian fratinon. Li ne plu marŝis sur la trotuaroj, sed flanke de ili, ĉefe se restis pluvakvo. Li tenis siajn manojn en la poŝoj kaj eltiris ilin kelkafoje por doni pugnobatojn facile kaj rapide. Li sciis vortojn, kiujn lia fratino ne komprenas. Li havis ofte truojn je siaj ŝtrumpoj kaj ankaŭ kelkafoje je siaj pantalonoj. Li diris ofte: ``Mia amiko Janko vidis tion'', aŭ ``mia amiko Delesar diris tion ĉi'', aŭ ``mia amiko Vuanzo faras nur mallertaĵojn". Li havis la manojn tute nigraj en la fino de la tago. Li ofte estis desegninta vizaĝetojn sur siaj ungoj. Li sciis paroli pri politiko; li diris "tiu malsprita ministro'', aŭ ``danĝera hundo, tiu deputato!''

Karlo nun havis ĉe la liceo multajn amikojn, aŭ pli bone multajn kamaradojn, ĉar li ne estimis ĉiujn egale. Estis Belnett, kiu multe parolis, kaj Pietro, kiu lin ĉiam aŭskultis. Estis Rigar, tre dika, kiun oni nomis ``kukurbo''; ĉe la banejo, li ne bezonis naĝi: tute senmove li restadis sur la akvo, kiel peco da ligno. Estis Paŭlo kaj Davido Rois, du fratoj, kiuj ĉiam tre bone laboris kaj dum la lecionoj neniam parolis aŭ bruis.

Estis ``Kokido", la malgrasa knabeto, ĉiam tre zorge vestita kaj ĝentila, sed ĉiam pala kaj timema; Rigar diris pri li: "Mi ne volus lin tuŝi, ĉar mi timus lin disrompi.'' Estis Kahn, filo de riĉa bankisto. Estis Donel, kiu ĉiam aĉetis bombonojn kaj sukeraĵojn. Estis Laminde, kiu tre bone deklamis; Servetti, kiu admirinde imitis ĉiujn bestoblekojn; Cenar, tre malavara; Holder, ĉiam pensema kaj silenta; Vuanzo, ĉiam ridanta; Robert, fiera kaj kolerema; Peter, filo de panisto, ĉiam kun ruĝa kravato; Man, kiu fabrikis fajfilojn.

Estis Pergo, kiu ĉiam iris paroli kun la profesoroj post la lecionoj. Estis Delasar, kiu pretendis esti kuzo de l'Prezidanto de la Franca Respubliko. Estis Tomaso, kun grandaj flikaĵoj ĉe la genuo sur siaj pantalonoj, kaj Gardiol, kiu tre bele ludis violonon kaj ricevis premion ĉe la muzika lernejo. Estis ankaŭ Stanen, ĉiam malbonodoranta; Belti kaj Travis, kiuj ambaŭ loĝis ekster la urbo. Estis Roĝers, kies patro havis aŭtomobilon. Allen estis Irlandano kaj Zerapumis estis Greko. Estis ankoraŭ kelkaj aliaj plue.

Sed la amiko de Karlo, la vera, la fidela, la plej bona, la plej lerta, estis Janko, kiu estis\ldots{}Janko.

\section*{Demandoj}

\emph{Rakontu skribe la supran ĉapitron.}

\begin{enumerate}
    \item  Ĉu Karlo sukcesis por la ekzameno?
    \item  Je kioma horo komenciĝis la lecionoj?
    \item  Ĉu Karlo havis multe da tempo por la tagmanĝo?
    \item  Kion li faris post kelkaj semajnoj?
    \item  Kiu estis la alnomo de Rigar?
    \item  Ĉu Holder estis parolema?
    \item  Kie loĝis Belti?
    \item  Kiu estis la plej fidela amiko de Karlo?
\end{enumerate}
