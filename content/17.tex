\chapter{Studento kaj poeto}

Ĉe l'fino de siaj gimnaziaj jaroj, Karlo bone sukcesis sian lastan ekzamenon. Lia patro decidis, ke li forveturu studadi en Heidelberg. Tie li pasigis unu jaron; sed malgraŭ la gajeco de l'studentoj kaj la beleco de l'urbeto, Karlo ne havis tre ĝojan vivadon. Li pli kaj pli fariĝis revema. Dum la aliaj drinkadis bieron kaj petoladis tra la stratoj, Karlo trankvile sidadis en sia ĉambro.

Li havis bonan brakseĝon, kaj, en la malvarmaj vesperoj, li tie sidis ĉe la fajro, pripensante aŭ legante. Li ofte verkis versaĵojn. Ankaŭ li skribis longajn leterojn al sia amiko Janko, kiu estis en Bruselo, studanta leĝosciencon. Karlo aŭskultis en Heidelberg kelkajn kursojn pri natura scienco, biologio kaj psikologio. Li decidis, ke proksiman jaron li komencos studadi medicinon. Por tio li ankaŭ veturis Bruselon, kiam estis finita lia jaro en Heidelberg.

En Bruselo, pli gaja estis por li la vivado. Tie li renkontis sian amikon Jankon kaj ankaŭ alian kunliceanon Laminde, kiu ĵus venis el Anglujo kaj studadis literaturon. La influo de Janko estis tre bona al Karlo, ĉar tiu amiko estis ĉiam energia kaj gaja. Li kondukadis Karlon al la juĝejoj, al politikaj kunvenoj, al sociaj kaj sciencaj paroladoj. Kiel antaŭe, li multe lernigis al sia revema amiko.

Laminde estis neriĉa junulo. Li devis perlabori por vivadi kaj studadi. Li donis lecionojn, li eĉ faris paroladojn pri la angla literaturo, li verkadis por gazetoj. Dank' al li, Janko kaj Karlo kelkafoje ricevis senpagajn teatrobiletojn, havigitajn de la redakcio de ĵurnalo, por kiu Laminde ofte skribis artikolojn.

Kiel Janko, Laminde multe kuraĝigis Karlon per sia vigleco. Kvankam li studadis medicinon, Karlo ĉiam pli interesiĝis je la literaturo kaj arto. Ambaŭ liaj amikoj tre incitis lin, ke li publikigu kelkajn el la poemoj, kiujn li verkis. Sed Karlo dum longa tempo estis tro timema. Fine li elektis tridek el ili kaj eldonis elegantan volumeton kun modernarta kovrilo kaj longe serĉita titolo: ``Oraj Flugiloj''. La unua versaĵo estis la jena:

\newpage

\begin{verse}
\textbf{Lasta flugo}

Vi kien flugas, papilio,\\
\vin Tremante kaj rapide?\\
Jam mortis rozo kaj lilio,\\
\vin Kaj venas frost' perfide.

Flugiloj viaj kvazaŭ lampo\\
\vin Briletas en nebulo:\\
Vi kion serĉas tra la kampo,\\
\vin Perdita somerulo?

Dezertaj estas la ĝardenoj\\
\vin Kaj vana l'amo via.\\
Skeletoj ŝajnas la vervenoj\\
\vin Sub laŭbo senfolia.

Vi kial flugas en malvarmo\\
\vin Tra flora la tombejo?\\
Sur ĉiu loko pluva larmo\\
\vin Nun restas en herbejo.

\newpage

Silentas ĉiuj en la nestoj,\\
\vin Kaj svenis bonodoroj;\\
Plu ne batalos vi kun vespoj\\
\vin Pri la floretaj koroj.

Jen restas nur en kampo vasta\\
\vin Velkinta krizantemo,\\
Sur kiu via kiso lasta\\
\vin Mortigos vin en tremo.
\end{verse}

\newpage

\section*{Demandoj}

\begin{enumerate}
    \item  Al kiu urbo veturis Karlo?
    \item  Kiom da monatoj li restis tie?
    \item  Nomu la tagojn de la semajno kaj la monatojn de la jaro.
    \item  Al kiu ofte skribis Karlo?
    \item  Kion li studis speciale?
    \item  Nomu kelkajn sciencojn kaj artojn.
    \item  Ĉu Karlo ŝatis skribi poeziojn?
    \item  Donu kelkajn vortojn rimantajn kun ``vento''.
    \item  Ĉu vi preferas prozon aŭ versojn?
\end{enumerate}
